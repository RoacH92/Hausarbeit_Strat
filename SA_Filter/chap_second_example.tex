\chapter{Ein zweites Beispiel Kapitel}

In diesem Kapitel könnten wir doch mal eine Tabelle, eine Aufzählung und ein paar Formeln einsetzen.

Fangen wir mit einer sehr einfachen Tabelle an. 

\begin{table}[h!]
  \centering
  \caption{Short Table}  % define title
  \begin{tabular}{l|c||r}
    1 & 2 & 3\\
    \hline
    a & b & c\\
  \end{tabular}
    \label{tab:easyTab}% don't forget to label!
\end{table}

Für den, der es gerne etwas komplexer mag, gibt es auch ein  Marko für Excel \href{https://www.ctan.org/pkg/excel2latex?lang=de}{\enquote{Excel2LaTex}} \autocite{MarkroExcel}, welches es erlaubt Tabellen aus Excel direkt in \LaTeX Code umzuwandeln und als Datei oder als Kopie über die Zwischenablage in euer Dokument einzufügen. Die unten stehende Tabelle wurde auf diesem Weg erzeugt. Wenn ihr kein Excel nutzen wollt, oder nicht zur Verfügung habt, dann habt ihr mit dem  \href{http://www.tablesgenerator.com}{Tabellengenerator} \autocite{TabGen} die Möglichkeit Tabellen online zu erstellen und den \LaTeX Code in euer Dokument einzufügen. 

% Table generated by Excel2LaTeX from sheet 'Tabelle1'
\begin{table}[!htbp]
  \centering
  \caption{More complex Table} % define title
    \begin{tabular}{crrrc} \toprule
    Test num. & \multicolumn{3}{c}{Testcase} & Mean Value\\ \cmidrule{2-4}
              &  Jan & Shashi & Sahana       &           \\ \midrule
        1     &  123 & 456    & 678          &   419     \\
        2     & 1235 & 235    & 2121         &   1197    \\ \midrule
        Sum   & 1358 & 691    & 2799         &   1616    \\ \bottomrule 
	\end{tabular}%
  \label{tab:complexTab}% don't forget to label!
\end{table}%

\newpage

Und Formeln gehen so...
\begin{equation}
		B > \frac{1}{n} \sum_{i=1}^{n}x_{i}
\end{equation}

nach lesen kann man hier \href{http://www.ma.tum.de/foswiki/pub/Ferienkurse/WiSe0809/LaTeX/2_Mathematik_print.pdf}{\hl{Formel Link TUM}}
\autocite{FormelTUM}
wie es gehen kann. Und hier gibt's einen Formeleditor für die, etwas sagen wir, effizienteren unter euch: \href{http://www.zahlen-kern.de/editor/}{http://www.zahlen-kern.de/editor/}
 
und so kann man auch noch Aufzählungen machen

\begin{enumerate}
	\item 
	und hier steht jetzt auch noch was
	\item 
	und hier sowieso
    \item und hier steht wie es geht
    \item
    \href{http://mirrors.ibiblio.org/CTAN/info/translations/enumitem/de/enumitem-de.pdf}{Anpassen von Listen mit dem \enquote{enumitem} Paket}
		\begin{enumerate}
			\item 
		\end{enumerate}
\end{enumerate}






\autocite{koma-script}
