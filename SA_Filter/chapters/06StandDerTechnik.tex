\chapter{Stand der Technik Filterüberwachung}
\label{ch:stand}
    Ziel dieses Kapitels ist einen Überblick über die unterschiedlichen genutzten Technologien zu bieten, die im bearbeiteten Themekomplex den momentanen Stand der Technik darstellen. In Bezug auf Gebäudetechnik ist diese Vorstellung jedoch kritisch zu sehen, da sich die Gebäudetechnik in den letzten Jahrzehnten im Zuge der Digitalisierung, und auch unter Gesichtspunkten wie Energieeffizienz, massiv weiterentwickelt hat. Dies gilt allerdings nur für Neubauten bzw. aufwendige Renovierungsmaßnahmen. Analog hierzu ist in älteren Gebäuden auch entsprechend alte Lüftungs- bzw. Klimatechnik verbaut, welche nicht immer mit modernen Regelungstrategien gesteuert wird, oder gar mit Bus-basierter \ac{glt} ausgestattet ist.
    \section{Direkte Warnsysteme}
    Unter dem Stichwort "Filterbruchüberwachung" lassen sich eine Vielzahl Lösungen finden (s. z.B. \cite{fbu}), um das Versagen unterschiedlicher Filter zu melden. Hierbei wird meist der triboelektrische Effekt genutzt.
    Dieser erzeugt eine zunehmende Messspannung, je mehr Partikel auf den Fühler an der Reingasseite treffen. Wird ein bestimmter Schwellenwert erreicht, wird ein Alarm über die \ac{glt} ausgelöst.
    Diese Art der Überwachung ist als zusätzliche Sicherheitsmaßnahme bei kritischen Anlagenbereichen oder z.B. der Filtration von toxischen Stoffen einzuordnen.
    Eine derartige Überwachung ist im Sinne einer prognostizierenden Überwachung nicht nützlich, da sie systembedingt erst Aufschluss über den Filterzustand geben kann, wenn bereits ein Schaden vorliegt.
    \section{Überwachungsstrategien}
    Üblicherweise wird eine Überwachung der Druckdifferenz am Filter, über eine Differenzmessung mit Schläuchen zur Reingas- und Rohgasseite durchgeführt. Die Messwerte werden dann entweder direkt über die \ac{glt} an eine zentrale Steuerung übertragen, oder zunächst an \ac{ddc}-Stationen auf Feldebene weiterverarbeitet und anschließend Warnungen bzw. Messwerte übertragen. Die Grenzwerte, ob in Feldstation oder Zentrale hinterlegt, basieren hierbei auf einer Mischung aus Herstellerangaben, Erfahrungswerten und Auslegungsrechnungen bei der Installation. 
    \section{Problematik und Grenzen}
    Da momentane Überwachungsstrategien auf der Einhaltung von Grenzwerten basieren, aber hierbei nicht mögliche Umweltfaktoren im Betrieb einbeziehen, lässt sich davon ausgehen, dass grobe Überschlagsrechnungen genutzt werden. Eine detailliertere Auslegung würde den Planungsaufwand nicht rechtfertigen, und auch Versuchsreihen, und hierfür nötige Infrastruktur (z.B. Labore) erforderlich machen. Die resultierenden Kosten stehen nicht im Verhältnis zu den Kosten einer relativ konservativen Auslegung. Dies hat zur Folge, dass die Mehrheit der Filter weit vor Ablauf ihres eigentlichen Lebenszeitendes getauscht werden. Außerdem wird ein Wartungsauftrag zum Tausch in der Regel erst dann ausgelöst, wenn festgelegte Grenzwerte überschritten werden. Selbstverständlich lassen sich hierbei Vorlaufzeiten für Auftragsabwicklung berücksichtigen, im Gegensatz zu prädiktiven Wartungsansätzen ist der Planungs-/Kommunikationsaufwand hierbei jedoch höher. Mit einer prädiktiven Wartung würden sich solche Aufträge hingegen mit ausreichend Vorlaufzeit planen lassen, was auch eventuelle Kosten bei der Abwicklung senkt, da diese z.B. an Schichtpausen, ohne Störung der laufenden Produktion, terminiert werden können.
    \section{Predictive Maintenance}
    \label{sec:predmain}
    Die Instandhaltung von Anlagen ist stets gekoppeltet an den Planungsaufwand, und damit verbundenen Kosten. Weitere Kostenfaktoren können z.B. die Kosten durch die Verschwendung bei der präventiven Instandhaltung sein. Hierbei werden Verschleißteile provisorisch an Hand eines Wartungsplans ausgetauscht.
    Die historische Ausgangslage ist daher die reaktive und präventive Instandhaltung, die auf Erfahrungswerten bzw. Auslegungsberechnungen beruht, wobei bei der reaktiven Instandhaltung nur dann gewartet wird wenn ein Versagen auftritt, und diese somit die einfachste Art der Instandhaltung darstellt. Mit der Einführung der Computer-basierten Überwachung von Maschinendaten erfolgte eine Revolution in der Instandhaltung. Diese Stufe stellt den Ist-Zustand aktueller industrieller Instandhaltungsstrategien dar. Ein sog. \ac{cms} bündelt hierbei Informationen zu Betriebsdaten einzelner Komponenten, analoge Aufgaben in der Gebäudetechnik erfüllt die \ac{glt} mit zentralem Server. Dieses Überwachungssystem wird kontinuierlich mit Messdaten aus permanenten Sensoren versorgt. Somit kann der Zustand der Anlage bzw. Maschine überwacht werden, um aus den gewonnenen Daten Kennwerte zu bilden. Auf Grundlage dieser Kennwerte können dann Instandhaltungsmaßnahmen abgeleitet werden. Dabei ist es möglich die Instandhaltung teilweise zu automatisieren, indem z.B. Grenzwerte für einzelne  Schlüsselparameter festgelegt werden. Hierzu müssen auch Wechselwirkungn einzelner Subsysteme miteinander berücksichtigt werden, was den Aufwand einer Einführung erhöht. Aktueller Stand ist daher die Prognose. Bei dieser Entwicklungsstufe werden statistische Analyseverfahren und Simulationen eingesetzt, um Fehlermuster zu erkennen und Ausfälle vorhersagen zu können. \cite{inst} 
    Der Aufwand steigt hierbei mit der Komplexität der Systeme extrem an, da allgemeingültige Aussagen schwierig zu treffen sind. Gutes Beispiel hierfür ist diese Arbeit, in der kein allgemeingültiges Modell für Luftfilter erstellbar ist, sonder nur für wenige, spezifische Luftfilter. 
    Bei dieser Entwicklungsstufe ist bereits der Einsatz von KI zur Mustererkennung und zur nachgelagerten Prognose auf Basis von Zustandsdaten denkbar.
    Zukünftige Instandhaltungssysteme werden soweit vernetzt sein, dass sie mit Hilfe von Schnittstellen zur Produktionsplanung selbst Instandhaltungsmaßnahmen präventiv auf Grundlage der prognostizierten Auslastung anfordern bzw. planen können. Hierbei wird dann zusätzlich die erwartbare Auslastung einkalkuliert. Eine automatisierte Erstellung von Wartungsaufträgen und ein Versenden dieser an externe Dienstleister ist ebenfalls denkbar. Alleinstellungsmerkmal ist hierbei die dezentrale Interpretation und Analyse von Sensordaten, und zunehmende Befähigung der Systeme auf Feldebene zur Netzwerkkommunikation.  Siehe auch \ref{sec:iiot}. Eine Adaption dieser Entwicklungsstufe auf \ac{lta}'s erscheint weniger sinnvoll, da diese mehr oder weniger kontinuierlich betrieben werden. Die dezentral Verfügbare Rechenleistung bietet jedoch, gerade im Hinblick auf moderne und energieeffiziente Machine Learning Algorithmen, Potenzial zur individuellen Überwachung von Filtern auf Feldebene.
    Denkbar wäre es, an Daten aus Versuchen auf Prüfständen mit einzelnen Filtern angelernte, KI-Modelle in \ac{ddc}-Stationen zu integrieren.
