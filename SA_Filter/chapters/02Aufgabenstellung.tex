\chapter{Aufgabenstellung}
\label{ch: Aufgabenstellung}
Im Rahmen dieser Studienarbeit soll eine Übersicht zu den möglichen Verschleißarten von industriellen Luftfiltern erstellt werden. In einem zweiten Schritt soll identifiziert werden, wie sich diese auftretenden Verschleißvorgänge messtechnisch erfassen lassen. Dabei ist insbesondere herauszuarbeiten, wie ein Verschleiß bei verschiedenen Fahrweisen einer Lüftungsanlage unter verschiedenen Umgebungsbedingungen zuverlässig identifiziert werden kann.
\section{Einarbeitung und Eingrenzung}
Von dieser Aufgabe ausgehend ergeben sich somit folgende Fragestellungen. Über diese wird zunächst der Umfang behandelter Filter eingegrenzt, und anschließend vorbereitend auf das technische Konzept messbare Parameter zur Erkennung von Verschleiß identifiziert.
\begin{itemize}
    \item Auswahl von mind. 3 Arten mechanischer Filter
    \item Verschleißarten in Luftfiltern verschiedener Bauweisen
    \item Identifikation geeigneter Messgrößen für die Erkennung dieser Verschleißarten
    \item Bewertung des Einflusses verschiedener Fahrweisen auf den Messgrößenverlauf
    \item Bewertung von Umgebungsbedingungen auf den Messgrößenverlauf
    \item Schlussfolgerungen für ein geeignetes Messkonzept
    \item Auswahl möglicher Messtechnik
\end{itemize}
\section{Konzept zur technischen Umsetzung}
Aufbauend auf diese Vorbereitung werden verschiedene Lösungsansätze hinsichtlich der automatisierten Erkennung verschiedener Verschleißzustände ausgearbeitet. Abschließend wird ein Konzept zur KI-gestützen Überwachung generiert und mit Hilfe eines Simulationsmodells validiert.
\begin{itemize}
    \item Welche Ansätze finden heute bereits Anwendung
    \item Inwieweit wäre eine KI-gestütze Messdatenauswertung sinnvoll
    \item Entwickeln eines Konzeptes für die technische Implementierung, z.B. auf Basis von DTL mit der Software KNIME
\end{itemize}