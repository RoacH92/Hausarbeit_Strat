\chapter{Einleitung}
\label{ch: Einleitung}
In konventionellen Anlagen zur Luftreinigung bzw. Staubabscheidung werden eine Vielzahl Filter unterschiedlicher Bauweisen eingesetzt. Die eigentlichen Filterelemente unterliegen allerdings fast immer einem gewissen Verschleiß, welcher durch diverse Umweltfaktoren begünstigt wird. Auch hygienische Aspekte wie Geruchsbelästigung oder Keimbildung an Filtermedien können als eine Art Verschleiß gewertet werden. Entsprechend sind für die meisten Filter beim Einbau festgelegte Standzeiten vorgeschrieben, insbesondere bei Anlagen für die Lüftung von z.B. Bürogebäuden oder Produktionsstätten. Die Kernfrage dieser Arbeit ist daher, ob es möglich ist diese festgelegten Standzeiten zu verlängern und gleichzeitig die Anlagensicherheit zu erhöhen, wenn man die Filter mit Hilfe vernetzter Sensorik überwacht, sowie die laufenden Messdaten KI-gestützt analysiert. Die vorliegende Arbeit hat hierbei einen Studiencharakter und versucht die Frage zu beantworten, ob eine KI-basierte Auswertung der Daten in diesem Kontext sinnhaft ist. Weiterhin wird ein technisches Konzept zur Umsetzung eines solchen Systems vorgeschlagen.  
    \section{Ausgangssituation}
    In der Klima- bzw. Lüftungstechnik werden schon lange Regelkreise eingesetzt, um beispielsweise den Volumenstrom zu einzelnen Räumen zu steuern bzw. konstant zu halten. Ergo ist in aller Regel bereits eine Sensorik zur Messung diverser Größen vorhanden. Moderne Gebäudetechnik setzt, im Zuge der zunehmenden Digitalisierung, bei der Verknüpfung der einzelnen Subsysteme solcher Anlagen bereits Bussysteme ein. Hierauf aufbauend existieren außerdem zustandsbasierte Überwachungssysteme für \ac{lta}'s, welche bei der Wartungsplanung unterstützen, indem z.B. bei Überschreiten einer Druckdifferenz eines Filters über einen Grenzwert, eine Warnung ausgegeben wird. Diese Grenzwerte sind naturgemäß mit relativ konservativen Sicherheiten beaufschlagt und berücksichtigen nicht alle Umweltparameter. Eine Prognose der Auslastung in Folge geänderter oder prognostizierbarer Umweltbedingungen erfolgt somit nicht. Resultierend werden Filter oftmals vor dem eigentlichen Ende ihrer Lebenszeit getauscht, was eine Form vermeidbarer Verschwendung darstellt.
    \section{Problemstellung}
    Der Verschleiß von Filtern ist von diversen Einflussparametern aus Umgebung, Last und Korrosionsvorgängen geprägt. Zusätzlich existieren unzählige unterschiedliche Filtertypen, was allgemeine Aussagen zum Filterverschleiß unmöglich macht. Allein im Anwendungsbereich Luftfilter werden Gasfilter, Membranfilter, elektrostatische Filter, Massekraftabscheider usw. eingesetzt, welche auf ebenso diversen Wirkmechanismen beruhen und somit auch von unterschiedlichen Verschleißarten betroffen sind. \newline 
    Eine modellhafte Prognose von Filterverschleiß ist also nur auf Grundlage von genauen Kenntnissen über Umgebung (insb. Staublast) und Anlage, sowie umfangreichen Versuchsdaten mit dem exakten Filtermodell möglich. Zusätzlich würde dies einen enormen Entwicklungs- und Planungsaufwand für jede einzelne \ac{lta} erfordern. Liegen jedoch historische Zeitreihendaten zu den genannten Parametern ausreichend feingranular vor und wurde zudem eine Bewertung der getauschten Filter durchgeführt und dokumentiert, wäre hier der Einsatz von \ac{ki} denkbar. Mit der Bildung von quantifizierbaren Kennwerten aus den genannten Daten kann so eine Aussage über den Verschleißzustand eines ungeprüften Filters auf Grundlage historischer Daten erfolgen. Die gewonnenen Kennwerte lassen sich dann als Wahrscheinlichkeiten für eine bestimmte Versagensart, und der \ac{ki}-Einsatz als statistisches Werkzeug verstehen.
    \section{Methodische Vorgehensweise}
    Da die vorherig genannten Daten zum Filterverhalten nur aus aufwendigen Versuchen generierbar und zudem standortspezifisch sind, liegt es nahe, diese zunächst modellhaft in einer Simulation zu erzeugen, um dem Rahmen einer Studienarbeit gerecht zu werden. Hierzu müssen zwangsläufig Annahmen über die Wirkung der Parameter auf die Filtereigenschaften und entsprechende Zusammenhänge getroffen werden. Es werden hierbei, soweit möglich, Versuchsauswertungen von Arbeiten anderer Verfasser zum Thema Filterverschleiß hinzugezogen. Insbesondere im Feld der US-amerikanischen Nuklearwissenschaften wurden hierzu Untersuchungen an \ac{hepa}-Filtern durchgeführt. In der Filtertechnik Branche hat sich zudem über den langen Bestand der Industrie hinweg ein äußerst konservativer Ansatz bei der Auslegungsrechnung durchgesetzt. Hierdurch sind hinreichende Unterschungen, auch wegen der hohen Diversität der unterschiedlichen Filter, selten zu finden, und teilweise auch schlicht überflüssig. Beispielsweise ist die Untersuchung des Berstdrucks von Filtern nicht national oder international genormt, stattdessen wird die Nutzung der Norm zur Bestimmung der Berstfestigkeit von Papier DIN EN ISO 2758 \cite{2758} empfohlen. \newline
    Zur Generierung von Daten werden in einer Simulation unterschiedliche Modelle implementiert, um die häufigste Schadensart zu modellieren. Anschließend kann eine nachgelagerte Analyse der baugleichen Filter mit Hilfe von KI und anhand der in der Simulation erreichten Lebensdauer erfolgen. Ziel ist hierbei eine Ausgabe der Ausfallwahrscheinlichkeit in einem definierten Zeitraum. Eine versuchgestütze Validierung der Simulation entfällt hierbei.\newline
    Während die Simulation hierzu mit Matlab/Simulink erstellt wird, erfolgt die \ac{ki}-gestütze Analyse der Daten mit der Software \ac{KNIME}. Das trainierte Modell auf Grundlage von \ac{DTL} wird dann wiederum mit einem neuen Datensatz aus der Simulation mit variierten Eingangsgrößen validiert. Abschließend wird das Potenzial einer KI-basierten Analyse von Luftfiltern an einem Filter eines Beispielsystems aufgezeigt.
    Die Gliederung der Arbeit folgt der Aufgabenstellung (s. Kap. \ref{ch: Aufgabenstellung}), wobei zunächst theoretisches Grundlagenwissen (s. Kap. \ref{ch:Theoretische Grundlagen}), welches zum Verständnis der Arbeit beiträgt, vorgestellt wird. Anschließend wird ein Beispielsystem vorgestellt, und ein Filter für die weitergehende Betrachtung ausgewählt (s. Kap. \ref{ch:auswahl}). Weiterhin werden unterschiedliche Schadensarten an Luftfiltern vorgestellt, sowie für die Erkennung dieser Schadensarten relevante Messgrößen ermittelt (s. Kap. \ref{ch:Filterverschleiß}). Es folgt eine Vorstellung des aktuellen Technikstands im Kontext der Filterüberwachung, wobei auch auf unterschiedliche Möglichkeiten zur Wartungsplanung eingegangen wird (s. Kap. \ref{ch:stand}). Hiervon ausgehend wird ein Grundkonzept vorgestellt, wie eine moderne Filterüberwachung aussehen könnte, die eine KI-basierte Überwachung ermöglicht (s. Kap. \ref{ch:konzept}). Ausgehend von der Simulation der Verläufe der vorherig erwähnten Messgrößen wird dann eine KI trainiert und getestet (s. Kap. \ref{ch:validierung}). Die Ergebnisse werden abschließend im letzten Kapitel (s. \ref{ch:schluss}) vorgestellt, und bewertet.